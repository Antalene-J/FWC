\documentclass[12pt,-letter paper]{article}
\usepackage{siunitx}
\usepackage{setspace}
\usepackage{gensymb}
\usepackage{xcolor}
\usepackage{caption}
%\usepackage{subcaption}
\doublespacing
\singlespacing
\usepackage[none]{hyphenat}
\usepackage{amssymb}
\usepackage{relsize}
\usepackage[cmex10]{amsmath}
\usepackage{mathtools}
\usepackage{amsmath}
\usepackage{commath}
\usepackage{amsthm}
\interdisplaylinepenalty=2500
%\savesymbol{iint}
\usepackage{txfonts}
%\restoresymbol{TXF}{iint}
\usepackage{wasysym}
\usepackage{amsthm}
\usepackage{mathrsfs}
\usepackage{txfonts}
\let\vec\mathbf{}
\usepackage{stfloats}
\usepackage{float}
\usepackage{hyperref}
\usepackage{cite}
\usepackage{cases}
\usepackage{subfig}
%\usepackage{xtab}
\usepackage{longtable}
\usepackage{multirow}
%\usepackage{algorithm}
\usepackage{amssymb}
%\usepackage{algpseudocode}
\usepackage{enumitem}
\usepackage{mathtools}
%\usepackage{eenrc}
%\usepackage[framemethod=tikz]{mdframed}
\usepackage{listings}
%\usepackage{listings}
\usepackage[latin1]{inputenc}
%%\usepackage{color}{   
%%\usepackage{lscape}
\usepackage{textcomp}
\usepackage{titling}
\usepackage{hyperref}
%\usepackage{fulbigskip}   
\usepackage{tikz}
\usepackage{graphicx}
\usepackage{circuitikz}
%\usepackage[left=1in, right=2in, top=1in, bottom=1in]{geometry}

\lstset{
  frame=single,
  breaklines=true
}
\let\vec\mathbf{}
\usepackage{enumitem}
\usepackage{graphicx}
\usepackage{siunitx}
\let\vec\mathbf{}
\usepackage{enumitem}
\usepackage{graphicx}
\usepackage{enumitem}
\usepackage{tfrupee}
\usepackage{amsmath}
\usepackage{amssymb}
\usepackage{mwe} % for blindtext and example-image-a in example
\usepackage{wrapfig}
\graphicspath{{figs/}}
\newcommand{\myvec}[1]{\ensuremath{\begin{pmatrix}#1\end{pmatrix}}}
\newcommand{\mydet}[1]{\ensuremath{\begin{vmatrix}#1\end{vmatrix}}}
\providecommand{\cbrak}[1]{\ensuremath{\left\{#1\right\}}}
\providecommand{\brak}[1]{\ensuremath{\left(#1\right)}}
\providecommand{\sbrak}[1]{\ensuremath{{}\left[#1\right]}}
\providecommand{\norm}[1]{\left\lVert#1\right\rVert}
\providecommand{\abs}[1]{\left\vert#1\right\vert}
\providecommand{\brak}[1]{\ensuremath{\left(#1\right)}}
\title{2016 12th}

\begin{document}

\maketitle{Questions}


A $4$-bit shift register circuit configured for right-shift operation, i.e.\\ $D_{in} \rightarrow A, A \rightarrow B, B \rightarrow C, C \rightarrow D$, is shown. If the present state of the shift register is $ABCD = 1101$, the number of clock cycles required to reach the state $ABCD = 1111$ is

\hfill (GATE-EC 2017,44)
\begin{figure}[H]
    \centering
	\begin{circuitikz}
	%nodes
	\draw (-1,0) node[xor port, rotate=180,scale =1.5](xor1) {};
	\draw (2,-3) node[rectangle, scale = 2,draw, outer sep = 0] (a) {A};
	\draw (3,-3) node[rectangle, scale = 2,draw, anchor = west, outer sep = 0] (b) at (a.east) {B};
	\draw (4,-3) node[rectangle, scale = 2,draw, anchor = west, outer sep = 0] (c) at (b.east) {C};
	\draw (5,-3) node[rectangle, scale = 2,draw, anchor = west, outer sep = 0] (d) at (c.east) {D};
        
        %connection
        \draw (xor1.out) to (-2,0)
        (-2,0) to (-2,-2.75)
        (-2,-2.75) to (1.4,-2.75);
        \draw (xor1.in 1) to (2,-0.43)
        (2,-0.43) to (2,-2.45);
        \draw (1.4,-3.25) to (0.25,-3.25);
        \draw (xor1.in 2) to (5.5,0.43)
        (5.5,0.43) to (5.5,-2.45) ;
        
        %naming
        \draw (-1,-2.5) node[rectangle, scale = 1.5]{$\text{D}_{\text{in}}$};
        \draw (0.25,-3.5) node[rectangle, scale = 1.5] {Clock};
        
\end{circuitikz}


\end{figure}



\end{document}
