\documentclass[12pt,-letter paper]{article}
\usepackage{siunitx}
\usepackage{setspace}
\usepackage{gensymb}
\usepackage{xcolor}
\usepackage{caption}
%\usepackage{subcaption}
\doublespacing
\singlespacing
\usepackage[none]{hyphenat}
\usepackage{amssymb}
\usepackage{relsize}
\usepackage[cmex10]{amsmath}
\usepackage{mathtools}
\usepackage{amsmath}
\usepackage{commath}
\usepackage{amsthm}
\interdisplaylinepenalty=2500
%\savesymbol{iint}
\usepackage{txfonts}
%\restoresymbol{TXF}{iint}
\usepackage{wasysym}
\usepackage{amsthm}
\usepackage{mathrsfs}
\usepackage{txfonts}
\let\vec\mathbf{}
\usepackage{stfloats}
\usepackage{float}
\usepackage{hyperref}
\usepackage{cite}
\usepackage{cases}
\usepackage{subfig}
%\usepackage{xtab}
\usepackage{longtable}
\usepackage{multirow}
%\usepackage{algorithm}
\usepackage{amssymb}
%\usepackage{algpseudocode}
\usepackage{enumitem}
\usepackage{mathtools}
%\usepackage{eenrc}
%\usepackage[framemethod=tikz]{mdframed}
\usepackage{listings}
%\usepackage{listings}
\usepackage[latin1]{inputenc}
%%\usepackage{color}{   
%%\usepackage{lscape}
\usepackage{textcomp}
\usepackage{titling}
\usepackage{hyperref}
%\usepackage{fulbigskip}   
\usepackage{tikz}
\usepackage{graphicx}
%\usepackage[left=1in, right=2in, top=1in, bottom=1in]{geometry}

\lstset{
  frame=single,
  breaklines=true
}
\let\vec\mathbf{}
\usepackage{enumitem}
\usepackage{graphicx}
\usepackage{siunitx}
\let\vec\mathbf{}
\usepackage{enumitem}
\usepackage{graphicx}
\usepackage{enumitem}
\usepackage{tfrupee}
\usepackage{amsmath}
\usepackage{amssymb}
\usepackage{mwe} % for blindtext and example-image-a in example
\usepackage{wrapfig}
\graphicspath{{figs/}}
\newcommand{\myvec}[1]{\ensuremath{\begin{pmatrix}#1\end{pmatrix}}}
\newcommand{\mydet}[1]{\ensuremath{\begin{vmatrix}#1\end{vmatrix}}}
\providecommand{\cbrak}[1]{\ensuremath{\left\{#1\right\}}}
\providecommand{\brak}[1]{\ensuremath{\left(#1\right)}}
\providecommand{\sbrak}[1]{\ensuremath{{}\left[#1\right]}}
\providecommand{\norm}[1]{\left\lVert#1\right\rVert}
\providecommand{\abs}[1]{\left\vert#1\right\vert}
\providecommand{\brak}[1]{\ensuremath{\left(#1\right)}}
\title{2016 12th}

\begin{document}

\maketitle{Questions}

\begin{enumerate}

\section{Vectors}
	\item If vectors $\overrightarrow{a}$ and $\overrightarrow{b}$ are such that
 $\mydet{\overrightarrow{a}} = \frac{1}{2}$, $\mydet{\overrightarrow{b}} = \frac{4}{\sqrt{3}}$
 and $\mydet{\overrightarrow{a} \times \overrightarrow{b}} = \frac{1}{\sqrt{3}}$, then find 
 $\mydet{\overrightarrow{a}.\overrightarrow{b}}$.

	\item If $\overrightarrow{a}$ and $\overrightarrow{b}$ are unit vectors, then what is the angle between 
$\overrightarrow{a}$ and $\overrightarrow{b}$ for $\overrightarrow{a} - \sqrt{2}\overrightarrow{b}$ to be an unit vector ?
	
	\item Find the distance between the planes 
		\begin{align}
			\overrightarrow{r}.\myvec{2\hat{i}-3\hat{j}+6\hat{k} } - 4 =0
		\end{align}
	and 
		\begin{align}
			\overrightarrow{r}.\myvec{6\hat{i}-9\hat{j} +18\hat{k}} +30 =0
		\end{align}

	\item Given that vectors $\overrightarrow{a}$, $\overrightarrow{b}$, $\overrightarrow{c}$ form a triangle such that 
$\overrightarrow{a} = \overrightarrow{b}+\overrightarrow{c}$. Find $p$, $q$, $r$, $s$ such that area of triangle is $5\sqrt{6}$ where $\overrightarrow{a} = p\hat{i} +q\hat{j}+r\hat{k}$, 
$\overrightarrow{b} = s\hat{i} +3\hat{j}+4\hat{k}$ and $\overrightarrow{c}=3\hat{i} +\hat{j}-2\hat{k}$.
	
	\item Find the co-ordinates of the point where the line $\overrightarrow{r}=(-\hat{i}-2\hat{j}-3\hat{k})+\lambda(3\hat{i} +4\hat{j}+3\hat{k})$ meets the plane which is perpendicular to the vector $\overrightarrow{n}=\hat{i}+\hat{j} +3\hat{k}$ and at a distance of
$\frac{4}{\sqrt{11}}$ from origin.

	
	\item Write the sum of intercepts cut off by the plane $\overrightarrow{r}.\myvec{2\hat{i}+\hat{j}-\hat{k}} - 5 = 0$ on the three axes.

	\item Find $\lambda$ and $\mu$ if
	\begin{align}
		\myvec{\hat{i} + 3\hat{j} + 9\hat{k}} \times \myvec{3\hat{i} - \lambda \hat{j} + \mu \hat{k}} = \overrightarrow{0}.
	\end{align}

	\item If $\overrightarrow{a} = 4\hat{i} - \hat{j} +\hat{k}$ and $\overrightarrow{b} = 2\hat{i} - 2\hat{j} + \hat{k}$, then find a unit vector parallel to the vector $\overrightarrow{a}+\overrightarrow{b}$.

	\item Find the equation of the plane which contains the line of intersection of the planes
	\begin{align}
		\overrightarrow{r}.\myvec{\hat{i} - 2\hat{j} + 3\hat{k}} - 4 &= 0 \text{  and}\\
		\overrightarrow{r}.\myvec{-2\hat{i} + \hat{j} + \hat{k}} + 5 &= 0
	\end{align}
and whose intercept on $x$-axis is equal to that of on $y$-axis.



\section{Linear Forms}
	\item Find the equation of plane passing through the points $A\myvec{3 & 2 & 1}$,
$B\myvec{4 & 2 & -2}$ and $C\myvec{6 & 5 & -1}$ and hence find the value of$\lambda$ for which 
$A\myvec{3 & 2 & 1}$, $B\myvec{4 & 2 & -2}$, $C\myvec{6 & 5 & -1}$ and $D\myvec{\lambda & 5 & 5}$
are coplanar.

	\item Find the equation of the plane containing two parallel lines
$\frac{x-1}{2} = \frac{y+1}{-1} = \frac{z}{3}$ and $\frac{x}{4} = \frac{y-2}{-2} = \frac{z+1}{6}$
Also, find if the plane thus obtained contains the line 
$\frac{x-2}{3} =\frac{y-1}{1} = \frac{z-2}{5}$ or not.



	\item For what value of $k$, the system of linear equations 
	\begin{align}
		x+y+z &= 2\\
		2x+y-z &=3\\
		3x+2y+kz &=4
	\end{align}
	has a unique solution? 
	
	\item Show that the four points $A(4,5,1)$, $ B(0,-1,-1)$, 
$C(3,9,4)$ and $D(-4,4,4)$ are coplanar.

	\item Find the coordinates of the foot of perpendicular drawn from the point
$A(-1, 8, 4)$ to the line joining the points $B(0, -1, 3)$ and $C(2,-3,-1)$. Hence
find the image of the point $A$ in the line $BC$.

	

\section{Differentiation}
	\item Differentiate $(\sin 2x)^x + \sin^{-1} \sqrt{3x}$ with respect to $x$.
	
	\item Differentiate $\tan^{-1} \brak{\frac{\sqrt{1 + x^2}-\sqrt{1-x^2}}{\sqrt{1+x^2}+\sqrt{1-x^2}}}$
with respect to $\cos^-1 x^2$.

	\item Determine the intervals in which the function $f (x) = x^4 - 8x^3 + 22x^2 - 24x+21$ is strictly increasing or strictly decreasing.

	\item Find the maximum and minimum values of $f (x) = \sec x + \log \cos^2 x$, $0 < x < 2\pi$.

	\item Find the eqaution of normal to the curve $ay^2 = x^3$ at the point whose $x$ coordinate is $am^2$
	
	
	\item If $\cos(a+y) = \cos y$ then prove that \\
$\frac{dy}{dx} = \frac{\cos^{2}(a+y)}{\sin a}$. 
Hence show that \\
$\sin a \frac{d^{2}y}{dx^{2}} + \sin 2(a+y)\frac{dy}{dx} = 0 $.

	\item Find $\frac{dy}{dx}$ if $y = \sin^{-1}\sbrak{\frac{6x - 4\sqrt{1-4x^2}}{5}}$ 

	\item Find the equation of the tangents to the curve $y = x^3 + 2x - 4$ which are perpendicular to line $x + 14y + 3 = 0$
	
	\item Show that semi-vertical angle of a cone of maximum volume and given slant height is
$\cos^{-1}\left( \frac{1}{\sqrt{3}} \right)$

	\item Prove that $ y = \frac{4\sin \theta}{2+ \cos \theta} - \theta$ ia an increasing function of $\theta$ on $\sbrak{0,\frac{\pi}{2}}$.



\section{Martrices}
	\item If $A$ is a square matrix such that $\mydet{A} = 5$, write the value of 
$\mydet{AA^{\text{T}}}$

	\item $A = \myvec{1 & 2 \\ 3 & -1}$ and $B = \myvec{1 & -4 \\ 3 & -2}$, find $\mydet{AB}$.

	\item If $A = \myvec{0 & 3 \\ 2 & -5}$ and $KA = \myvec{0 & 4a \\ -8 & 5b}$ find the values of $k$ and $a$.

	\item Ishan wants to donate a rectangular plot of land for a school in his village. When he was asked to give dimensions of the plot, he told that if its length is decreased by $50m$ and breadth is increased by $50m$, then its area will remain same, but if length is decreased by $10m$ and breadth is decreased by $20m$, then its area will decrease by $5300m^2$. Using matrices, find the dimensions of the plot. Also give reason why he wants to donate the plot for a school.

	\item Using the properties of determinants, prove that:
	\begin{align}
		\mydet{(b+c)^2 & a^2 & bc \\
		(c+a)^2 & b^2 & ca\\
		(a+b)^2 & c^2 & ab} = (a - b) (b-c) (c-a) (a+b+c) (a^2 + b^2 + c^2)
	\end{align}
	
	\item Using elementary row operations, find the inverse of the following matrix :
	\begin{align}
		A = \myvec{2 & -1 & 3\\
		-5 & 3 & 1 \\
		-3 & 2 & 3}
	\end{align}



	\item If $A = \myvec{\cos \alpha & \sin \alpha\\ -\sin \alpha & \cos \alpha}$, find $\alpha$ satisfying $0<\alpha<\frac{1}{2}$ when $A + A^{\text{T}} = \sqrt{2}I_{2}$, where $A^{\text{T}}$ is transpose of $A$

	\item If $A$ is a $3\times3$ matrix and $\mydet{3A} = k \mydet{A}$ then write the value of $k$

	\item Using properties of determinants, prove that
	\begin{align}
		\mydet{(x + y)^2 & zx & zy \\
		zx & (z+y)^2 & xy \\
		zy & xy & (z+x)^2}
		 = 2xyz (x + y + z)^3
	\end{align}
	
	\item If 
	\begin{align}
		A = \myvec{1 & 0 & 2\\
		0 & 2 & 1 \\
		2 & 0 & 3}
	\end{align}
and $A^3-6A^2+7A+kI_3=0$ find $k$.


\section{Integration}
	\item Find: $\int \frac{1- \sin x}{\sin x (1 + \sin x)} dx$

	\item Find: $\int \sbrak{\log (\log x)+ \frac{1}{(\log x)^2}} dx$

	\item Evaluate: $\int_0^\frac{\pi}{2} \frac{\sin^2x}{\sin x + \cos x} dx$

	\item Evaluate : $ \int_0^1 \cot^{-1}\brak{1 - x + x^2} dx$

	\item Solve the differential equation: $(x + 1) \frac{dy}{dx} - y = e^{3x} (x + 1)^3$ 

	\item Solve the differential equation : $2y e^{\frac{x}{y}}dx + \brak{y - 2x e^{\frac{x}{y}}}dy = 0$

	\item Using integration find the area of the region ${(x, y) : y^2 \leq 6ax \text{ and }
 x^2+y^2 \leq 16a^2}$.



	\item Find : $\int \frac{(2x-5)e^{2x}}{(2x-3)^3} dx$

	\item Find : $\int \frac{x^2 +x +1}{(x^2 + 1)(x + 2)} dx$

	\item Evaluate : $\int_{-2}^{2} \frac{x^2}{1+5^x} dx$

	\item Find : $\int (x+3)\sqrt{3 - 4x - x^2} dx$\\

	\item Find the particular solution of difference equation :\\
 $\frac{dy}{dx} = - \frac{x + y\cos x}{1 + \sin x}$ \\
 given that $y = 1$ when $x = 0$.

	\item Find the particualr solution of the differential equation
	\begin{align}
		2y e^{\frac{x}{y}} dx + (y -2x e^{\frac{x}{y}}) dy = 0
	\end{align}
given that $x = 0$ when $y = 1$.

	\item Using the method of integration, find the area of the triangular region whose vertices are $(2, 2)$, $(4, 3)$ and $(1, 2)$.


\section{Function}
	\item Find $k$, if 
	\begin{align}
		f(x) = \begin{cases} k\sin \frac{\pi}{2}(x+1) &,x \leq 0\\
		\frac{\tan x - \sin x}{x^3} &, x>0
		\end{cases}
	\end{align}
is continous at $x=0$
	
	\item Let $f: \text{N} \rightarrow \text{N}$ be a function defined as 
	$f(x) = 4x^2 + 12x + 15.$
Show that $f: \text{N} \rightarrow \text{S}$ is invertible (where S is range of $f$).
Find the inverse of $f$ and hence find $f^{-1}(31)$ and $f^{-1}(87)$.


	\item If 
	\begin{align}
		f(x) = 
		\begin{cases}
			\frac{\sin(a+1)x + 2\sin}{x} &,x<0 \\
			2 &,x = 0\\
			\frac{\sqrt{1+bx}-1}{x} &,x>0
		\end{cases}
	\end{align}
is continuous at $x = 0$, then find the values of $a$ and $b$.

	\item Let $A = R \times R$ and $*$ be a binary operation on $A$ defined by
	\begin{align}
		(a, b) * (c, d) = (a + c, b + d)
	\end{align}
	Show that $*$ is commutative and associative. Find the identity element for $*$
on $A$. Also find the inverse of every element $(a, b) \in A$.
	
	

\section{Probability}
	\item There are two bags $A$ and $B$. Bag $A$ contains $3$ white and $4$ red balls whereas bag $B$ contains $4$ white and $3$ red balls. Three balls are drawn at random (without replacement) from one of the bags and are found to be two white and one red. Find the probability that these were drawn from bag $B$.

	\item Three numbers are selected at random (without replacement) from first six positive integers. If $X$ denotes the smallest of the three numbers obtained, find the probability distribution of $X$. Also find the mean and variance of the distribution.


	\item A bag $X$ contains $4$ white balls and $2$ black balls, while another bag $Y$ contains
$3$ white balls and $3$ black balls. Two balls are drawn (without replacement) at
random from one of the bags and were found to be one white and one black.
Find the probability that the balls were drawn from bag $Y$.

	\item $A$ and $B$ throw a pair of dice alternately, till one of them gets a total of 10 and
wins the game. Find their respective probabilities of winning, if $A$ starts first.

	\item Three numbers are selected at random (without replacement) from first six
positive integers. Let $X$ denote the largest of the three numbers obtained. Find
the probability distribution of $X$. Also, find the mean and variance of the
distribution.


\section{Optimization}
	\item A diet is to contain at least $80$ units of Vitamin $A$ and $100$ units of minerals. 
Two foods $\text{F}_1$ and $\text{F}_2$ are available costing $5$ rupees per unit and $6$ rupees per unit respectively. 
One unit of food $\text{F}_1$ contains $4$ units of vitamin $A$ and $3$ units of minerals whereas
 one unit of food $\text{F}_2$ contains $3$ units of vitamin $A$ and $6$ units of minerals. 
 Formulate this as a linear programming problem. Find the minimum cost of diet that consists of mixture of these two foods and also meets minimum nutritional requirement.
 
 
 	\item  A retired person wants to invest an amount of $50,000$ rupees. His broker recommends investing in two type of bonds $A$ and $B$ yielding $10\%$ and $9\%$ return respectively on the invested amount. He decides to invest at least
$20,000$ rupees in bond $A$ and at least $10,000$ rupees in bond $B$. He also wants to invest at least as much in bond $A$ as in bond $B$. Solve this linear programming problem graphically to maximise his returns.

	\item A typist charges $145$ rupees for typing $10$ English and $3$ Hindi pages, while charges for typing $3$ English and $10$ Hindi pages are $180$ rupees. Using matrices, 
find the charges of typing one English and one Hindi page separately. 
However typist charged only $2$ rupees per page from a poor student Shayam for $5$ Hindi pages.
How mcuh less was charged from ths poor boy? which values are reflected in this problem?

	

\section{Algebra}
	\item Prove that $ 2\sin^{-1} \brak{\frac{3}{5}} - \tan^{-1} \brak{\frac{17}{31}} = \frac{\pi}{4}$

	\item Solve the equation for $x$: $\cos (\tan^{-1}) = \sin \brak{\cot^{-1} \frac{3}{4}} = \sin \brak{\cot^{-1} \frac{3}{4}}$

	
	\item Solve for 
	\begin{align}
		x: \tan^{-1}(x-1) + \tan^{-1}x + \tan^{-1}(x+1) = \tan^{-1}3x
	\end{align}

	\item Prove that 
	\begin{align}
	\tan^{-1} \brak{\frac{6x-8x^{3}}{1-12x^{2}}} - \tan^{-1} \brak{\frac{4x}{1-4x^{2}}} = \tan^{-1}2x;
	\end{align}
	\begin{align}
		\mydet{2x} < \frac{1}{\sqrt{3}}
	\end{align}

	
\end{enumerate}
\end{document}

