\let\negmedspace\undefined
\let\negthickspace\undefined
\documentclass[journal,12pt,onecolumn]{IEEEtran}
\usepackage{gensymb}
\usepackage{amssymb}
\usepackage[cmex10]{amsmath}
\usepackage{amsthm}
\usepackage[export]{adjustbox}
\usepackage{bm}
\usepackage{longtable}
\usepackage{enumitem}
\usepackage{mathtools}
\usepackage[breaklinks=true]{hyperref}
\usepackage{listings}
\usepackage{color}                                            %%
\usepackage{array}                                            %%
\usepackage{longtable}                                        %%
\usepackage{calc}                                             %%
\usepackage{multirow}                                         %%
\usepackage{hhline}                                           %%
\usepackage{ifthen}                                           %%
\usepackage{lscape}     
\usepackage{multicol}
% \usepackage{enumerate}
\DeclareMathOperator*{\Res}{Res}
\renewcommand\thesection{\arabic{section}}
\renewcommand\thesubsection{\thesection.\arabic{subsection}}
\renewcommand\thesubsubsection{\thesubsection.\arabic{subsubsection}}
\renewcommand\thesectiondis{\arabic{section}}
\renewcommand\thesubsectiondis{\thesectiondis.\arabic{subsection}}
\renewcommand\thesubsubsectiondis{\thesubsectiondis.\arabic{subsubsection}}

\hyphenation{op-tical net-works semi-conduc-tor}
\def\inputGnumericTable{}                                 %%
\lstset{
frame=single, 
breaklines=true,
columns=fullflexible
}
%To reomve number from section
\setcounter{secnumdepth}{1}
%To remove number from align
\makeatletter
\renewcommand\tagform@[1]{}
\makeatother
%increase spcae between item
\setlength{\itemsep}{8pt}
\begin{document}
\newtheorem{theorem}{Theorem}[section]
\newtheorem{problem}{Problem}
\newtheorem{proposition}{Proposition}[section]
\newtheorem{lemma}{Lemma}[section]
\newtheorem{corollary}[theorem]{Corollary}
\newtheorem{example}{Example}[section]
\newtheorem{definition}[problem]{Definition}
\newcommand{\BEQA}{\begin{eqnarray}}
\newcommand{\EEQA}{\end{eqnarray}}
\newcommand{\define}{\stackrel{\triangle}{=}}
\bibliographystyle{IEEEtran}
\providecommand{\mbf}{\mathbf}
\providecommand{\pr}[1]{\ensuremath{\Pr\left(#1\right)}}
\providecommand{\qfunc}[1]{\ensuremath{Q\left(#1\right)}}
\providecommand{\sbrak}[1]{\ensuremath{{}\left[#1\right]}}
\providecommand{\lsbrak}[1]{\ensuremath{{}\left[#1\right.}}
\providecommand{\rsbrak}[1]{\ensuremath{{}\left.#1\right]}}
\providecommand{\brak}[1]{\ensuremath{\left(#1\right)}}
\providecommand{\lbrak}[1]{\ensuremath{\left(#1\right.}}
\providecommand{\rbrak}[1]{\ensuremath{\left.#1\right)}}
\providecommand{\cbrak}[1]{\ensuremath{\left\{#1\right\}}}
\providecommand{\lcbrak}[1]{\ensuremath{\left\{#1\right.}}
\providecommand{\rcbrak}[1]{\ensuremath{\left.#1\right\}}}
\theoremstyle{remark}
\newtheorem{rem}{Remark}
\newcommand{\sgn}{\mathop{\mathrm{sgn}}}
\providecommand{\abs}[1]{\left\vert#1\right\vert}
\providecommand{\res}[1]{\Res\displaylimits_{#1}} 
\providecommand{\norm}[1]{\left\lVert#1\right\rVert}
%\providecommand{\norm}[1]{\lVert#1\rVert}
\providecommand{\mtx}[1]{\mathbf{#1}}
\providecommand{\mean}[1]{E\left[ #1 \right]}
\providecommand{\fourier}{\overset{\mathcal{F}}{ \rightleftharpoons}}
%\providecommand{\hilbert}{\overset{\mathcal{H}}{ \rightleftharpoons}}
\providecommand{\system}{\overset{\mathcal{H}}{ \longleftrightarrow}}
	%\newcommand{\solution}[2]{\textbf{Solution:}{#1}}
\newcommand{\solution}{\noindent \textbf{Solution: }}
\newcommand{\cosec}{\,\text{cosec}\,}
\providecommand{\dec}[2]{\ensuremath{\overset{#1}{\underset{#2}{\gtrless}}}}
\newcommand{\myvec}[1]{\ensuremath{\begin{pmatrix}#1\end{pmatrix}}}
\newcommand{\mydet}[1]{\ensuremath{\begin{vmatrix}#1\end{vmatrix}}}
\numberwithin{equation}{subsection}
\makeatletter
\@addtoreset{figure}{problem}
\makeatother
\let\StandardTheFigure\thefigure
\let\vec\mathbf
\renewcommand{\thefigure}{\theproblem}
\def\putbox#1#2#3{\makebox[0in][l]{\makebox[#1][l]{}\raisebox{\baselineskip}[0in][0in]{\raisebox{#2}[0in][0in]{#3}}}}
     \def\rightbox#1{\makebox[0in][r]{#1}}
     \def\centbox#1{\makebox[0in]{#1}}
     \def\topbox#1{\raisebox{-\baselineskip}[0in][0in]{#1}}
     \def\midbox#1{\raisebox{-0.5\baselineskip}[0in][0in]{#1}}
\vspace{3cm}
\title{CBSE MATHEMATICS 2016}
\author{065 Set 1 N\\
Antalene J (FWC22246)}
\maketitle
\bigskip
\renewcommand{\thefigure}{\theenumi}
\renewcommand{\thetable}{\theenumi}

\section{Section-A}
\textbf{Qestion numbers 1 to 6 carry 1 mark each}\\
\renewcommand{\theequation}{\theenumi}
\begin{enumerate}[label=\thesection.\arabic*.,ref=\thesection.\theenumi]
\numberwithin{equation}{enumi}
\item If $A = \myvec{\cos \alpha & \sin \alpha\\ -\sin \alpha & \cos \alpha}$, find $\alpha$ satisfying $0<\alpha<\frac{1}{2}$ when $A + A^{\text{T}} = \sqrt{2}I_{2}$, where $A^{\text{T}}$ is transpose of A

\item If $A$ is a $3\times3$ matrix and $\mydet{3A} = k \mydet{A}$ then write the value of k

\item For what value of $k$, the system of linear equations 
	\begin{align}
		x+y+z &= 2\\
		2x+y-z &=3\\
		3x+2y+kz &=4
	\end{align}
	has a unique solution? 

\item Write the sum of intercepts cut off by the plane $\overrightarrow{r}.\myvec{2\hat{i}+\hat{j}-\hat{k}} - 5 = 0$ on the three axes.

\item Find $\lambda$ and $\mu$ if
	\begin{align}
		\myvec{\hat{i} + 3\hat{j} + 9\hat{k}} \times \myvec{3\hat{i} - \lambda \hat{j} + \mu \hat{k}} = \overrightarrow{0}.
	\end{align}

\item If $\overrightarrow{a} = 4\hat{i} - \hat{j} +\hat{k}$ and $\overrightarrow{b} = 2\hat{i} - 2\hat{j} + \hat{k}$, then find a unit vector parallel to the vector $\overrightarrow{a}+\overrightarrow{b}$.

%new section
\section{Section-B}
\textbf{Qestion numbers 7 to 23 carry 4 mark each}\\
\item Solve for 
	\begin{align}
		x: \tan^{-1}(x-1) + \tan^{-1}x + \tan^{-1}(x+1) = \tan^{-1}3x
	\end{align}

\item Prove that 
	\begin{align}
	\tan^{-1} \brak{\frac{6x-8x^{3}}{1-12x^{2}}} - \tan^{-1} \brak{\frac{4x}{1-4x^{2}}} = \tan^{-1}2x;
	\end{align}
	\begin{align}
		\mydet{2x} < \frac{1}{\sqrt{3}}
	\end{align}

\item A typist charges 145 rupees for typing 10 English and 3 Hindi pages, while charges for typing 3 English and 10 Hindi pages are 180 rupees. Using matrices, 
find the charges of typing one English and one Hindi page separately. 
However typist charged only 2 rupees per page from a poor student Shayam for 5 Hindi pages.
How mcuh less was charged from ths poor boy? which values are reflected in this problem?

\item If 
	\begin{align}
		f(x) = 
		\begin{cases}
			\frac{\sin(a+1)x + 2\sin}{x} &,x<0 \\
			2 &,x = 0\\
			\frac{\sqrt{1+bx}-1}{x} &,x>0
		\end{cases}
	\end{align}
is continuous at $x = 0$, then find the values of a and b.

\item If $\cos(a+y) = \cos y$ then prove that \\
$\frac{dy}{dx} = \frac{\cos^{2}(a+y)}{\sin a}$. 
Hence show that \\
$\sin a \frac{d^{2}y}{dx^{2}} + \sin 2(a+y)\frac{dy}{dx} = 0 $.

\item Find $\frac{dy}{dx}$ if $y = \sin^{-1}\sbrak{\frac{6x - 4\sqrt{1-4x^2}}{5}}$ 

\item Find the equation of the tangents to the curve $y = x^3 + 2x - 4$ which are perpendicular to line $x + 14y + 3 = 0$

\item Find : $\int \frac{(2x-5)e^{2x}}{(2x-3)^3} dx$

\item Find : $\int \frac{x^2 +x +1}{(x^2 + 1)(x + 2)} dx$

\item Evaluate : $\int_{-2}^{2} \frac{x^2}{1+5^x} dx$

\item Find : $\int (x+3)\sqrt{3 - 4x - x^2} dx$\\

\item Find the particular solution of difference equation :\\
 $\frac{dy}{dx} = - \frac{x + y\cos x}{1 + \sin x}$ \\
 given that $y = 1$ when $x = 0$.

\item Find the particualr solution of the differential equation
	\begin{align}
		2y e^{\frac{x}{y}} dx + (y -2x e^{\frac{x}{y}}) dy = 0
	\end{align}
given that $x = 0$ when $y = 1$.

\item Show that the four points $A(4,5,1)$, $ B(0,-1,-1)$, 
$C(3,9,4)$ and $D(-4,4,4)$ are coplanar.

\item Find the coordinates of the foot of perpendicular drawn from the point
$A(-1, 8, 4)$ to the line joining the points $B(0, -1, 3)$ and $C(2,-3,-1)$. Hence
find the image of the point $A$ in the line $BC$.

\item A bag $X$ contains 4 white balls and 2 black balls, while another bag $Y$ contains
3 white balls and 3 black balls. Two balls are drawn (without replacement) at
random from one of the bags and were found to be one white and one black.
Find the probability that the balls were drawn from bag $Y$.

\item $A$ and $B$ throw a pair of dice alternately, till one of them gets a total of 10 and
wins the game. Find their respective probabilities of winning, if $A$ starts first.


%new section
\section{Section-C}
\textbf{Qestion numbers 24 to 35 carry 4 mark each}\\

\item Three numbers are selected at random (without replacement) from first six
positive integers. Let $X$ denote the largest of the three numbers obtained. Find
the probability distribution of $X$. Also, find the mean and variance of the
distribution.

\item Let $A = R \times R$ and $*$ be a binary operation on $A$ defined by
	\begin{align}
		(a, b) * (c, d) = (a + c, b + d)
	\end{align}
Show that $*$ is commutative and associative. Find the identity element for $*$
on $A$. Also find the inverse of every element $(a, b) \in A$.

\item Prove that $ y = \frac{4\sin \theta}{2+ \cos \theta} - \theta$ ia an ncreasing function of $\theta$ on $\sbrak{0,\frac{\pi}{2}}$.

\item Show that semi-vertical angle of a cone of maximum volume and given slant height is
$\cos^{-1}\left( \frac{1}{\sqrt{3}} \right)$

\item Using the method of integration, find the area of the triangular region whose vertices are $(2, 2)$, $(4, 3)$ and $(1, 2)$.

\item Find the equation of the plane which contains the line of intersection of the planes
	\begin{align}
		\overrightarrow{r}.\myvec{\hat{i} - 2\hat{j} + 3\hat{k}} - 4 &= 0 \text{  and}\\
		\overrightarrow{r}.\myvec{-2\hat{i} + \hat{j} + \hat{k}} + 5 &= 0
	\end{align}
and whose intercept on $x$-axis is equal to that of on $y$-axis.

\item  A retired person wants to invest an amount of 50,000 rupees. His broker recommends investing in two type of bonds $A$ and $B$ yielding 10\% and 9\% return respectively on the invested amount. He decides to invest at least
20,000 rupees in bond $A$ and at least 10,000 rupees in bond $B$. He also wants to invest at least as much in bond $A$ as in bond $B$. Solve this linear programming problem graphically to maximise his returns.

\item Using properties of determinants, prove that
	\begin{align}
		\mydet{(x + y)^2 & zx & zy \\
		zx & (z+y)^2 & xy \\
		zy & xy & (z+x)^2}
		 = 2xyz (x + y + z)^3
	\end{align}
	
\item If 
	\begin{align}
		A = \myvec{1 & 0 & 2\\
		0 & 2 & 1 \\
		2 & 0 & 3}
	\end{align}
and $A^3-6A^2+7A+kI_3=0$ find $k$.

\end{enumerate}
\end{document}
